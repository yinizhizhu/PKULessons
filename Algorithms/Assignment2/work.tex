\documentclass[a4paper,11pt,UTF8]{ctexart}
\usepackage{setspace}
\usepackage{amsmath}
\usepackage{enumitem}
\usepackage{fancyhdr}
\usepackage{lastpage}%获得总页数
\usepackage{multirow}
\pagestyle{fancy}
\fancyhf{}
\lhead{学号:1701214050}
\chead{姓名:李佳河}
\rhead{}
\lfoot{}
\cfoot{}
\rfoot{\thepage / \pageref{LastPage}}

\title{算法课第2次作业}
\author{}
\date{}
\begin{document}
\begin{spacing}{1.2}
\setlength{\parsep}{0.9ex}

\maketitle

\setlength{\parindent}{0em}

作业得分(打印时请保留此项):

\thispagestyle{fancy}
\fancyhf{}
\lhead{学号:1701214050}
\chead{姓名:李佳河}
\rhead{}
\lfoot{}
\cfoot{}
\rfoot{\thepage / \pageref{LastPage}}


\begin{table}[!htb]
\centering
\begin{tabular}{|c|c|c|c|c|c|c|}
\hline
 & 题目1 & 题目2 & 题目3 & 题目4 & 题目5 & 总分\\
\hline
\multirow{2}{*}{分数} &  &   &  &  &  &  \\
 &  &   &  &  &  &  \\
\hline
\multirow{2}{*}{阅卷人}  &  &   &  &  &  &  \\
 &  &   &  &  &  &  \\
\hline
\end{tabular}
\end{table}

注意事项:
\begin{enumerate}[itemindent=2em]
\footnotesize
  \item 算法课作业均采用A4纸打印,\underline{\textbf{左上角装订}};
  \item 不需要复制题目内容,直接在每一题的标题下填写解题过程即可;
  \item 对于复杂公式或图形,可以留空白后手写完成,\textbf{文字部分应该打印};
  \item 注意填写页眉中的姓名、学号;
  \item 打印时请保留第一页上方的“作业得分”表格。
\end{enumerate}


\textbf{\large{题目1}}
\setlength{\parindent}{2em}

存在没有强不稳定因素的完美匹配

 算法中由男人主动追求女人。

算法步骤:
第一轮,每个男人都选择自己名单上排在首位的女人,并向她求婚。这种时候会出现两种情况:(1)该女人还没有被男人追求过,则该女人接受该男人的请求。(2)若该女人已经接受过其他男人的追求,那么该女人会将该男人与她的现任男友进行比较。若更喜欢她的男友,那么拒绝这个人的追求;否则,踢掉现任男友,与这个男人匹配。

第一轮结束后,有些男人已经有伴侣,有些男人仍然是单身。

在第二轮追求行动中,单身男人都从所有未拒绝过他的女人中,向自己最中意的那一个求婚,不管她现在是否是单身。按照上面所述方案进行匹配。直到所有人都不在是单身。

这个算法肯定能够得到稳定的婚姻:

\begin{enumerate}[itemindent=3em]
  \item 所有人都能配上对。  假如有一个男人单身,那么这个男人必定是向所有的女人求过婚。而女人只要被求过婚,就不可能是单身。也就是说此时所有的女人都不是单身,与有一个男人一直单身是相悖的。所以假设不成立。该算法最后所有人都能匹配上。

  \item 随着轮数的增加,男士追求的对象越来越糟,而女士的男友则可能变得越来越好。  假设男\textit{A}和女\textit{1}各有各自的对象,但是比起现在的对象,男\textit{A}更喜欢女\textit{1},所以,在此之前男\textit{A}肯定已经跟女\textit{1}表白过的,并且女\textit{1}拒绝了男\textit{A},也就是女\textit{1}有了比男\textit{A}更好的男友,不会出现强不稳定的情况。
\end{enumerate}


不存在没有弱不稳定因素的完美匹配

一个反例就是,有两个男人$m_1$和$m_2$,两个女人$w_1$和$w_2$。两个男人都认为两个女人在他心中都一样,而两个女人都更喜欢男人$m_1$。此时这种情况之下,两个可能的匹配都是存在弱不稳定因素的。所以,不存在没有弱不稳定因素的匹配的。





~\\

\setlength{\parindent}{0em}
\textbf{\large{题目2}}
\setlength{\parindent}{2em}

对于每条船的时间表,我们需要找到一个停靠港口,让这条船在这个月的剩下的时间一直停靠这个港口。也就是说,这个最后的停靠港口就是截断点。

每条船都有一个按时间排序的访问港口的时间表。我们依据每条船的时间表,获得每个港口的一个时间表:访问自己按逆序时间访问这个港口的船的时刻表。这里将船看作是男人,将港口看作是女人,利用题目1中的算法来求得完美匹配。并且,这个定下来的匹配将是一个满足题设条件“一个港口只能停靠一条船”。

假设这个匹配是不可行的,即会存在一个港口$p_k$已经停靠了一条船$s_j$后又有另外一条船$s_i$将要途经港口$p_k$。而在我们的算法中,$s_i$“更喜欢(时间表上更早访问)”港口$p_k$,并且港口$p_k$“更喜欢(时间表上更晚访问)”$s_i$。很明显,在已经存在一个晚于$s_i$访问$p_k$,$s_i$不可能是$p_k$“更喜欢”的,说明假设是不成立的。所以,这个匹配是可行的。



~\\

\setlength{\parindent}{0em}
\textbf{\large{题目3}}
\setlength{\parindent}{2em}


$$f_2(n) < f_3(n) < f_6(n) < f_1(n) < f_4(n) < f_5(n)$$
$$ \sqrt{2n} < n + 10 < n^2 \log n < n^{2.5} < 10^n < 100^n $$


~\\
\setlength{\parindent}{0em}
\textbf{\large{题目4}}
\setlength{\parindent}{2em}

为了区分这两种蝴蝶,构造一个无向图\textit{G=(V, E)}。每一个蝴蝶都是一个点,而$v_i$和$v_j$之间的边表示这$v_i$和$v_j$有一个判断。这里,为了表示和说明方便,用\textit{typea}和\textit{typeb}代表两种不同的蝴蝶类型。同时,这个无向图\textit{G}并不需要是联通的。所以,对于给定的\textit{m}个判断,构造的结果可能会包含多个联通分量。

	具体的判断过程还需要标记各个与判断相关的节点,然后根据标记结果去和给出的m个判断进行对比。

	对于无向图\textit{G=(V, E)}的每一个联通分量$G_i$进行标记:任意选择一个节点$v_0$作为起始节点,并将$v_0$标记为\textit{typea}。然后利用广度优先的方式访问:如果接下来访问的节点$v_j$和上一层与$v_j$存在判断连接的$v_i$的判断是“相同类型”,那么就将$v_j$标记为$v_i$的类型;判断是“不同类型”,则将$v_j$标记为与$v_i$类型不同的另外一个类型。

	有了上面的标记,我们得到了与判断相关的节点的标记。现在我们重新审视一下给定的两个节点的判断是否符合最终的标记结果。如果判断相关的两个节点$v_i$和$v_j$是“相同类型”,而$v_i$和$v_j$的标记结果是不同的,则给出的判断是有矛盾的。如果判断相关的两个节点$v_i$和$v_j$是“不同类型”,而$v_i$和$v_j$标记的结果是相同的,则给出的判断是有矛盾的。如果判断所有给出的\textit{m}个判断都是没有出现过提到的这两个有矛盾的情形,则给出的\textit{m}个判断是没有矛盾的。



~\\

\setlength{\parindent}{0em}
\textbf{\large{题目5}}
\setlength{\parindent}{2em}

我们构造一个有向图图\textit{G},那么题设要求就变成了去检测\textit{G}是否是有环的。

对于每个人$p_i$,我们定义两个节点$b_i$(未知的出生日期)和$d_i$(死亡日期)。对于没有重叠部分的用\textit{($d_i$,$b_j$)}表示\textit{$p_i$}和\textit{$p_j$}的联系。对于有重叠的用\textit{($b_i$,$d_j$)}和\textit{($b_j$,$d_i$)}表示$p_i$和$p_j$是重叠的。就这样我们可以构建一个有向图\textit{G}。

现在,假设\textit{G}是有环的。有环就意味着这环中的每个节点都是先于下一个节点开始的,但是,我们无法找到一个最开始的节点。因此,这些信息不是完全正确的。

假设\textit{G}是无环的。这里就存在一个拓扑序。如果我们可以用这个顺序作为所有人的出生和死亡日期,就可以得到和给定信息一致的顺序。



\end{spacing}

\end{document} 